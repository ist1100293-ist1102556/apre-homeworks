\documentclass[12pt]{article}
\usepackage[paper=letterpaper,margin=2cm]{geometry}
\usepackage{amsmath,amssymb,amsfonts}
\usepackage{newtxtext, newtxmath}
\usepackage{enumitem}
\usepackage{titling}
\usepackage{nicematrix}
\usepackage[colorlinks=true]{hyperref}
\usepackage{graphicx}
\usepackage{listings}
\usepackage{mathtools}

\setlength{\droptitle}{-6em}

\DeclareMathOperator*{\argmax}{arg\,max}

\begin{document}

\newcommand{\prob}{\textrm{P}}
\newcommand{\ind}{\perp\!\!\!\!\!\perp} 
\newcommand{\notind}{\not\perp\!\!\!\!\!\perp}
\newcommand{\defeq}{\vcentcolon=}

\center
Aprendizagem 2023\\
Homework III -- Group 016\\
(ist1100293, ist1102556)\vskip 1cm

\large{\textbf{Part I}: Pen and paper}\normalsize

\begin{enumerate}[leftmargin=\labelsep]
    \item Perform one epoch of the EM clustering algorithm and determine the new parameters.
    Hint: we suggest you to use numpy and scipy, however disclose the intermediary results step by step.

    \paragraph{E-Step} First we will compute the posterior probabilities for each observation for each cluster, with the initial mixture:

    \begin{equation}
    \begin{aligned}
        P(C=c_k|\mathbf{x}) &\propto P(\mathbf{x}|C=c_k)P(C=c_k) \\
        &\propto P(Y_1 = y_1, Y_2 = y_2, Y_3= y_3|C=c_k)P(C=c_k) \\
        &\propto P(Y_1 = y_1|C=c_k)P(Y_2 = y_2, Y_3=y_3|C=c_k)P(C=c_k) \\
        &\propto p_k^{y_1}(1-p_k)^{1-y_1}N_k(y_2, y_3)\pi_k
    \end{aligned}
    \end{equation}

    \begin{equation}
    \begin{aligned}
        P(C=c_1|\mathbf{x}_1) &\propto p_1N_1(0.6, 0.1)\pi_1 &\qquad P(C=c_2|\mathbf{x}_1) &\propto p_2N_2(0.6, 0.1)\pi_2 \\
        &\propto 0.009986 &\qquad &\propto 0.041866 \\
        P(C=c_1|\mathbf{x}_1) &\approx \frac{0.009986}{0.009986+0.041866} &\qquad P(C=c_2|\mathbf{x}_1) &\approx \frac{0.041866}{0.009986+0.041866} \\
        &\approx 0.1926 &\qquad &\approx 0.8074
    \end{aligned}
    \end{equation}

    \begin{equation}
    \begin{aligned}
        P(C=c_1|\mathbf{x}_2) &\propto (1-p_1)N_1(-0.4, 0.8)\pi_1 &\qquad P(C=c_2|\mathbf{x}_2) &\propto (1-p_2)N_2(-0.4, 0.8)\pi_2 \\
        &\propto 0.017517 &\qquad &\propto 0.010229 \\
        P(C=c_1|\mathbf{x}_2) &\approx \frac{0.017517}{0.017517+0.010229} &\qquad P(C=c_2|\mathbf{x}_2) &\approx \frac{0.010229}{0.017517+0.010229} \\
        &\approx 0.6313 &\qquad &\approx 0.3687
    \end{aligned}
    \end{equation}

    \begin{equation}
    \begin{aligned}
        P(C=c_1|\mathbf{x}_3) &\propto (1-p_1)N_1(0.2, 0.5)\pi_1 &\qquad P(C=c_2|\mathbf{x}_3) &\propto (1-p_2)N_2(0.2, 0.5)\pi_2 \\
        &\propto 0.023931 &\qquad &\propto 0.019437 \\
        P(C=c_1|\mathbf{x}_3) &\approx \frac{0.023931}{0.023931+0.019437} &\qquad P(C=c_2|\mathbf{x}_3) &\approx \frac{0.019437}{0.023931+0.019437} \\
        &\approx 0.5518 &\qquad &\approx 0.4482
    \end{aligned}
    \end{equation}

    \begin{equation}
    \begin{aligned}
        P(C=c_1|\mathbf{x}_4) &\propto p_1N_1(0.4, -0.1)\pi_1 &\qquad P(C=c_2|\mathbf{x}_4) &\propto p_2N_2(0.4, -0.1)\pi_2 \\
        &\propto 0.008857 &\qquad &\propto 0.043575 \\
        P(C=c_1|\mathbf{x}_4) &\approx \frac{0.008857}{0.008857+0.043575} &\qquad P(C=c_2|\mathbf{x}_4) &\approx \frac{0.043575}{0.008857+0.043575} \\
        &\approx 0.1689 &\qquad &\approx 0.8311
    \end{aligned}
    \end{equation}

    \vskip 2cm

    Using the notation: $\gamma_{ki}=P(C = c_k |\mathbf{x}_i)$. We get:

    \begin{equation}
    \begin{aligned}
        \gamma_{11} = 0.1926 &\qquad \gamma_{21} = 0.8074 \\
        \gamma_{12} = 0.6313 &\qquad \gamma_{22} = 0.3687 \\
        \gamma_{13} = 0.5518 &\qquad \gamma_{23} = 0.4482 \\
        \gamma_{14} = 0.1689 &\qquad \gamma_{24} = 0.8311
    \end{aligned}
    \end{equation}

    \paragraph{M-Step} Now, we need to adjust the cluster parameters. For the normal distribuition, we have:

    \begin{equation}
        \mu_k = \frac{\sum_{i=1}^{n}\gamma_{ki}\mathbf{x}_i}{\sum_{i=1}^{n}\gamma_{ki}} \qquad \Sigma_k = \frac{\sum_{i=1}^{n}\gamma_{ki}(\mathbf{x}_i-\mu_i)(\mathbf{x}_i-\mu_i)^T}{\sum_{i=1}^{n}\gamma_{ki}}
    \end{equation}

    Where $\mathbf{x}_i$ are vectors with only $y_2$ and $y_3$. That gives us:

    \begin{equation}
    \begin{aligned}
        \mu_1 &= \begin{bmatrix}
            0.0265 \\ 0.5071
        \end{bmatrix} &\qquad \mu_2&= \begin{bmatrix}
            0.3091 \\ 0.2104
        \end{bmatrix} \\
        \Sigma_1 &= \begin{bmatrix}
            0.1414  & -0.1054 \\
            -0.1054 & 0.09605 
        \end{bmatrix} &\qquad \Sigma_2 &= \begin{bmatrix}
            0.1083  & -0.0887 \\
            -0.0887 &  0.1041
        \end{bmatrix}
    \end{aligned}
    \end{equation}

    To adjust the parameter of the Bernoulli distribuition, we will use the same estimator as the mean in a normal distribuition.

    \begin{equation}
        p_i = \frac{\sum_{i=1}^{n}\gamma_{ki}x_i}{\sum_{i=1}^{n}\gamma_{ki}}
    \end{equation}

    Where $x_i$ is the $y_1$ variable of the $\mathbf{x}_i$ observation. Which gives us:

    \begin{equation}
    \begin{aligned}
        p_1 = 0.2340 \qquad p_2 = 0.6673
    \end{aligned}
    \end{equation}

    Finally, for the priors, we have:

    \begin{equation}
        \pi_k = \frac{\sum_{i=1}^{N}\gamma_{ki}}{\sum_{j=1}^{N}\sum_{i=1}^{n}\gamma_{ji}}
    \end{equation}

    Which gives us:

    \begin{equation}
        \pi_1 = 0.3862 \qquad \pi_2 = 0.6138
    \end{equation}

    \item Given the new observation, $\mathbf{x}_{new} = \begin{pmatrix}
        1 \\ 0.3 \\ 0.7 \end{pmatrix}$, determine the cluster memberships (posteriors).

        \begin{equation}
            P(C = c_k | \mathbf{x}_{new}) \propto p_kN_k(0.3, 0.7)\pi_k
        \end{equation}

        \begin{equation}
        \begin{aligned}
            P(C = c_1 | \mathbf{x}_{new}) &= \frac{p_1N_1(0.3, 0.7)\pi_1}{p_1N_1(0.3, 0.7)\pi_1 + p_2N_2(0.3, 0.7)\pi_2} \\
            &\approx 0.0803 \\
            P(C = c_2 | \mathbf{x}_{new}) &= \frac{p_2N_2(0.3, 0.7)\pi_2}{p_1N_1(0.3, 0.7)\pi_1 + p_2N_2(0.3, 0.7)\pi_2} \\
            &\approx 0.9197 \\
        \end{aligned}
        \end{equation}
    
    \item Performing a hard assignment of observations to clusters under a ML assumption, identify
    the silhouette of the larger cluster under a Manhattan distance.

    Under a ML assumtion, we will ignore the priors, so:

    \begin{equation}
        P(C = c_k|\mathbf{x}) = P(\mathbf{x}|C = c_k) = p_k^{y_1}(1-p_k)^{y_1}N_k(y_2, y_3)
    \end{equation}

    Also, since we are doing hard assignments, we dont need to normalize these resutls.

    \begin{equation}
    \begin{aligned}
        P(C = c_1 | \mathbf{x}_1) &= 0.2315 &\qquad P(C = c_2 | \mathbf{x}_1) &= 0.9495 \\
        P(C = c_1 | \mathbf{x}_2) &= 1.2663 &\qquad P(C = c_2 | \mathbf{x}_2) &= 0.0887 \\
        P(C = c_1 | \mathbf{x}_3) &= 1.4381 &\qquad P(C = c_2 | \mathbf{x}_3) &= 0.4542 \\
        P(C = c_1 | \mathbf{x}_4) &= 0.0207 &\qquad P(C = c_2 | \mathbf{x}_4) &= 0.7233
    \end{aligned}
    \end{equation}

    That way, we can assign $\mathbf{x}_1$, $\mathbf{x}_4$ to $c_1$ and $\mathbf{x}_2$, $\mathbf{x}_3$ to $c_2$.

    Now, to compute the silhouette, we need to compute first the distance between all points:

    \begin{equation}
        d(\mathbf{x}, \mathbf{y}) = |y_1 - x_1| + |y_2 - x_2| + |y_3 - x_3|
    \end{equation}

    \begin{equation}
    \begin{aligned}
        d(\mathbf{x}_1, \mathbf{x}_2) &= 1 + 1   + 0.7 &= 2.7 \\
        d(\mathbf{x}_1, \mathbf{x}_3) &= 1 + 0.4 + 0.4 &= 1.8 \\
        d(\mathbf{x}_1, \mathbf{x}_4) &= 0 + 0.2 + 0.2 &= 0.4 \\
        d(\mathbf{x}_2, \mathbf{x}_3) &= 0 + 0.6 + 0.3 &= 0.9 \\
        d(\mathbf{x}_2, \mathbf{x}_4) &= 1 + 0.8 + 0.9 &= 2.7 \\
        d(\mathbf{x}_3, \mathbf{x}_4) &= 1 + 0.2 + 0.6 &= 1.8
    \end{aligned}
    \end{equation}

    Then, the silhouette of a single point will be:
    \begin{equation}
        s(\mathbf{x}_i) = 1 - \frac{a(\mathbf{x}_i)}{b(\mathbf{x}_i)}
    \end{equation}

    Where $a(\mathbf{x})$ is the average distance from $\mathbf{x}$ to the other members of it's cluster, and $b(\mathbf{x})$ is the maximum of the average distance to the members of other cluster.

    \begin{equation}
    \begin{aligned}
        s(\mathbf{x}_1) &= 1 - \frac{0.4}{\frac{1}{2}(2.7+1.8)} = 0.8222 \\
        s(\mathbf{x}_2) &= 1 - \frac{0.9}{\frac{1}{2}(2.7+2.7)} = 0.6667 \\
        s(\mathbf{x}_3) &= 1 - \frac{0.9}{\frac{1}{2}(1.8+1.8)} = 0.5 \\
        s(\mathbf{x}_4) &= 1 - \frac{0.4}{\frac{1}{2}(2.7+1.8)} = 0.8222
    \end{aligned}
    \end{equation}

    Now, the silhouette for the clusters is the average silhouette of it's members:

    \begin{equation}
    \begin{aligned}
        s(c_1) = \frac{s(\mathbf{x}_2) + s(\mathbf{x}_3)}{2} = 0.8222 \\
        s(c_2) = \frac{s(\mathbf{x}_1) + s(\mathbf{x}_4)}{2} = 0.5833
    \end{aligned}
    \end{equation}

    \item Knowing the purity of the clustering solution is 0.75, identify the number of possible classes
    (ground truth).

    With the purity definition:

    \begin{equation}
        purity = \frac{1}{n} \sum_{k=1}^{K} \max_j(|C_k \cap L_j|)
    \end{equation}

    By inputing the known values, we get:

    \begin{equation}
    \begin{aligned}
        \frac{1}{4}(\max_j(|C_1 \cap L_j|) + \max_j(|C_2 \cap L_j|)) &= 0.75 \\
        \max_j(|C_1 \cap L_j|) + \max_j(|C_2 \cap L_j|) &= 3
    \end{aligned}
    \end{equation}

    Since $|C_1|$ and $|C_2|$ are both $2$, both of the parcels will have a maximum value of 2. The parcels also will have a minimum value of $1$. That way, for the sum to equal $3$, one of them needs to be $1$ and the other $2$. That means that for one of the clusters, both of the observations need to belong to the same class, while for the other cluster, each observation needs to belong to a different class. That means that, if we assume that all classes are represented in the data, there can either be $2$ or $3$ distinct classes. That depends if the class of the two elements of the same cluster is also represented on the other cluster, or not.
\end{enumerate}

\vskip 10cm

\large{\textbf{Part II}: Programming and Critical Analysis}\normalsize


\end{document}
